% #######################################
% ########### FILL THESE IN #############
% #######################################
\def\mytitle{ BOOLEAN LOGIC IMPLEMENTATION  BY USING DFLIPFLOP  WITH ARDUINO ASSEMBLY}
\def\mykeywords{}
\def\myauthor{SATTHISH D}
\def\contact{sathishmahadevvarma.ds55@gmail.com}
\def\mymodule{}
% #######################################
% #### YOU DON'T NEED TO TOUCH BELOW ####
% #######################################
\documentclass[12pt, a4paper]{article}
\usepackage[a4paper,outer=1.5cm,inner=1.5cm,top=1.75cm,bottom=1.5cm]{geometry}
\twocolumn
\usepackage{graphicx}
\graphicspath{{./images/}}
\usepackage{karnaugh-map}
%colour our links, remove weird boxes
\usepackage[colorlinks,linkcolor={black},citecolor={blue!80!black},urlcolor={blue!80!black}]{hyperref}
%Stop indentation on new paragraphs
\usepackage[parfill]{parskip}
%% Arial-like font
\usepackage{lmodern}
\renewcommand*\familydefault{\sfdefault}
%Napier logo top right
\usepackage{watermark}
%Lorem Ipusm dolor please don't leave any in you final report ;)
\usepackage{circuitikz}
\usetikzlibrary{calc}
\usepackage{tikz}

\usetikzlibrary{shapes, arrows, chains, decorations.markings,intersections,calc}
\usepackage{lipsum}
\usepackage{xcolor}
\usepackage{listings}
%give us the Capital H that we all know and love
\usepackage{float}
%tone down the line spacing after section titles
\usepackage{titlesec}
%Cool maths printing
\usepackage{amsmath}
\usepackage{tabularx}
%PseudoCode
\usepackage{algorithm2e}

\titlespacing{\subsection}{0pt}{\parskip}{-3pt}
\titlespacing{\subsubsection}{0pt}{\parskip}{-\parskip}
\titlespacing{\paragraph}{0pt}{\parskip}{\parskip}
\newcommand{\figuremacro}[5]{
    \begin{figure}[#1]
        \centering
        \includegraphics[width=#5\columnwidth]{#2}
        \caption[#3]{\textbf{#3}#4}
        \label{fig:#2}
    \end{figure}
}

\lstset{
frame=single,
breaklines=true,
columns=fullflexible
}

\thiswatermark{\centering \put(-15,-100.0){\includegraphics[scale=0.3]{logo}} }
\title{\mytitle}
  \author{\myauthor\hspace{1em}\\\contact\\FWC220101    IITH-Future Wireless Communications     Assignment-1\hspace{0.5em}\hspace{0.5em}\mymodule}
\date{}
\hypersetup{pdfauthor=\myauthor,pdftitle=\mytitle,pdfkeywords=\mykeywords}
\sloppy
% #######################################
% ########### START FROM HERE ###########
% #######################################
 
 \begin{document}
 \maketitle
 \tableofcontents
    %\begin{figure}
        %\centering
        %\includegraphics[width=\linewidth]{}
        %\includegraphics[width=\linewidth]{}
    %    \caption{\textbf{Karnaugh Map}}
     %   \label{fig:my_label}
    %\end{figure}
  \textbf{}{\mykeywords}
\vspace{5mm}      
\section{Abstract}
This shows counter constructed with 3 D FLIPFLOPS  Implementation of boolean expressions by using arduino with assembly

\section{Way of using arduino }
in arduino we are having ports B,D,C.here we are using port D pin 2 is taken as output pin.port B pins 8,9,10 pins are taken as a inputs.pin 13 as clk
\hfill \break
\hfill \break
\vspace{5mm}

\vspace{5mm}     
\section{Components}
     
       \begin{tabularx}{0.43\textwidth}{
  | >{\centering\arraybackslash}X
  | >{\centering\arraybackslash}X
  | >{\centering\arraybackslash}X | }
\hline
\textbf{Component}&\textbf{Value}& \textbf{Quantity}\\ \hline
Arduino   & UNO & 1 \\ \hline
Bread board   & - & 1 \\ \hline
Jumper wires  & M-M & 22 \\ \hline
sevensegment           & - & 1\\ \hline
7447      &  & 1\\ \hline
7474      &  & 2\\ \hline
\end{tabularx}

\section{Truth table for given K-map}
\begin{tabularx}{0.50\textwidth} {
  | >{\centering\arraybackslash}X
  | >{\centering\arraybackslash}X
  | >{\centering\arraybackslash}X
  | >{\centering\arraybackslash}X
  | >{\centering\arraybackslash}X
  | >{\centering\arraybackslash}X
  | >{\centering\arraybackslash}X
  | >{\centering\arraybackslash}X 
  | >{\centering\arraybackslash}X  | }
  \hline
 X & Y & Z &X+ &Y+ & Z+ & D2 & D1 & D0\\
 0 & 0 & 0&0 & 0 & 1 &0 & 0 &1 \\  
 0 & 0 & 1&0 & 1 & 1 &0 & 1 &1\\ 
 0 & 1 & 1&0 & 1 & 0 &0 & 1 &0\\ 
 0 & 1 & 0&1 & 1 & 0 &1 & 1 &0\\ 
 1 & 1 & 0&1 & 1 & 1 &1 & 1 &1\\ 
 1 & 1 & 1&1 & 0 & 1 &1 & 0 &1\\ 
 1 & 0 & 1&1 & 0 & 0 &1 & 0 &0\\ 
 1 & 0 & 0&0 & 0 & 0 &0 & 0 &0\\ 
\hline
\end{tabularx}
\begin{center}
TABLE 1
\end{center}


\section{procedure}
\textbf{Step 1:} connect 5v of the Arduino to the top red of the bread board ang GND to the bottom green
\hfill \break
\hfill \break
\textbf{Step 2:} connect 2 7474 ics in  the bread board for futher connections.1 st flipflop act as 2 flipflops
\hfill \break
\hfill \break
\textbf{Step 3:} connect d13 pin in the arduino to connect breabboard the to connect the as clk to the flipflop. d13 pin connect to the pin3 and pin 11 of  1st flipflop and pin3 of 2nd flipflop 
\hfill \break
\hfill \break
\textbf{Step 4:} connect 5v to the pin 14 and pin 1 and pin4  and pin10 and pin 13 of the 1st flipflop and pin1 and pin 14 and pin4 of 2nd flipflop 
\hfill \break
\hfill \break
\textbf{Step 5:} connect 7447 decoder pins 13 to 9 to the sevensegment given a,b,c,d,e respectively and pin f to the 15 pin of sevensegment.com of the seven segment connect to vcc by using through resisitor  
\hfill \break
\hfill \break
\textbf{Step 6:} now pins 8,9,10 and 2,3,4 pins in ardino connect them in breadboard in parallel way
\hfill \break
\hfill \break
\textbf{Step 7:} take 3 new cables connect their one end in the end of the 3 parallel connected cables.other end of the cables connect to the 7,1,2 pins in the decoder serially and pin6 to the gnd
\hfill \break
\hfill \break
\textbf{Step 8:}finally run code in the arduino and check results of newmerical values shown sevensegment like 0,1,3,2,6,7,5,4 as per truthtable
\hfill \break
\hfill \break

\vspace{5mm}
\section{equation by truth table}
D2 have high logic(2,5,7,6)=sum(2,5,7,6)\\     \hfill \break
D1 have high logic(1,3,2,6)=sum(1,3,2,6)\\\hfill \break
D0 have high logic(0,1,6,7)=sum(0,1,6,7)\\
\hfill \break
\hfill \break

\section{circuit connections}
     
        \begin{center}
\begin{tabular}{ | m{5.0em} | m{1.0cm}| m{1.0cm} |m{1.0cm} |m{0.5cm} |m{1.0cm} |m{4.0cm} |m{3.0cm} | } 
  \hline
  \textbf{Arduino} & 2,4 & 3& 8,10& 9& 13& 5V& GND \\ 
  \hline
  \textbf{7474} & 2 & 12 & 5&9 &3, 11& 1, 4, 10, 13, 14& 7\\ 
  \hline
  \textbf{7447} & 7 & 1 & & & & 16& 2, 6, 8\\ 
  \hline
\end{tabular}
\end{center}

\section{BooleanEquation}
By solving the given K-map diagram we get the boolean equation as follows \\ 
D2 = AC + BC'\\
D1 = A'C +BC'\\
D0 = A'B' +AB\\
\section{Software}
\textbf{Execute the following code using the below provided link.}\\
\begin{center}
\fbox{\parbox{8.5cm}{\url{https://github.com/satthish-devaragatla}}}
\end{center}
 

 
%\begin{document}

\vspace{5mm}   


%\end{document}

\end{document}
