\documentclass[12pt]{article}
\usepackage{graphicx}
%\documentclass[journal,12pt,twocolumn]{IEEEtran}
\usepackage[none]{hyphenat}
\usepackage{graphicx}
\usepackage{listings}
\usepackage[english]{babel}
\usepackage{graphicx}
\usepackage{caption} 
\usepackage{hyperref}
\usepackage{booktabs}
\usepackage{commath}
\usepackage{gensymb}
\usepackage{array}
\usepackage{amsmath}   % for having text in math mode
\usepackage{listings}
\lstset{
  frame=single,
  breaklines=true
}
  
%Following 2 lines were added to remove the blank page at the beginning
\usepackage{atbegshi}% http://ctan.org/pkg/atbegshi
\AtBeginDocument{\AtBeginShipoutNext{\AtBeginShipoutDiscard}}
%


%New macro definitions
\newcommand{\mydet}[1]{\ensuremath{\begin{vmatrix}#1\end{vmatrix}}}
\providecommand{\brak}[1]{\ensuremath{\left(#1\right)}}
\providecommand{\norm}[1]{\left\lVert#1\right\rVert}
\newcommand{\solution}{\noindent \textbf{Solution: }}
\newcommand{\myvec}[1]{\ensuremath{\begin{pmatrix}#1\end{pmatrix}}}
\let\vec\mathbf


\begin{document}
\begin{center}
\title{\textbf{Properties of vectors}}
\date{\vspace{-5ex}} %Not to print date automatically
\maketitle
\end{center}
\setcounter{page}{1}
\section{12$^{th}$ Maths - Exercise 10.4.10}

\begin{enumerate}
\item The area of a parallelogram whose adjacent sides are represented by the vectors $\vec{a}=\hat{i}-\hat{j}+3\hat{k}\text{ and } \vec{b}=2\hat{i}-7\hat{j}+\hat{k}$
\section{Solution}
Now,
\begin{align}
\text{Let } \vec{A} = \myvec{1\\-1\\3} \text{ and } \vec{B} = \myvec{2\\ -7 \\ 1}\\
\end{align}
The cross product or vector product of $\vec{A},\vec{B}$ is defined as
\begin{align}
	\vec{A} \times \vec{B} = \myvec{\mydet{\vec{A}_{23}&\vec{B}_{23}\\\vec{A}_{31}&\vec{B}_{31}\\\vec{A}_{12}&\vec{B}_{12}}}
\end{align}
Hence
\begin{align}
	\mydet{\vec{A}_{23}&\vec{B}_{23}}&=\mydet{-1&-7\\3&1}=\myvec{-1+21}=20\\
	\mydet{\vec{A}_{31}&\vec{B}_{31}}&=\mydet{1&2\\3&1}=\myvec{1-6}=-5\\
	\mydet{\vec{A}_{12}&\vec{B}_{12}}&=\mydet{1&2\\-1&-7}=\myvec{-7+2}=-5\\
\end{align}

which can be represented in matrix form as
\begin{align}
	\vec{A} \times \vec{B}&=\myvec{20\\-5\\-5}.
\end{align}
Hence
\begin{align}
\norm{\vec{A} \times \vec{B}}&=\sqrt{20^2-5^2-5^2}\\
 &= 15\sqrt{2}\\
\end{align}

\end{enumerate}
\end{document}