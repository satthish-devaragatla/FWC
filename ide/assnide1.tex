% #######################################
% ########### FILL THESE IN #############
% #######################################
\def\mytitle{IMPLEMENTATION OF BOOLEAN LOGIC USING D FLIPFLOPS IN ARDUINO IDE}
\def\mykeywords{}
\def\myauthor{SATTHISH D}
\def\contact{sathishmahadevvarma.ds55@gmail.com}
\def\mymodule{}
% #######################################
% #### YOU DON'T NEED TO TOUCH BELOW ####
% #######################################
\documentclass[12pt, a4paper]{article}
\usepackage[a4paper,outer=1.5cm,inner=1.5cm,top=1.75cm,bottom=1.5cm]{geometry}
\twocolumn
\usepackage{graphicx}
\graphicspath{{./images/}}
\usepackage{karnaugh-map}
%colour our links, remove weird boxes
\usepackage[colorlinks,linkcolor={black},citecolor={blue!80!black},urlcolor={blue!80!black}]{hyperref}
%Stop indentation on new paragraphs
\usepackage[parfill]{parskip}
%% Arial-like font
\usepackage{lmodern}
\renewcommand*\familydefault{\sfdefault}
%Napier logo top right
\usepackage{watermark}
%Lorem Ipusm dolor please don't leave any in you final report ;)
\usepackage{circuitikz}
\usetikzlibrary{calc}
\usepackage{tikz}

\usetikzlibrary{shapes, arrows, chains, decorations.markings,intersections,calc}
\usepackage{lipsum}
\usepackage{xcolor}
\usepackage{listings}
%give us the Capital H that we all know and love
\usepackage{float}
%tone down the line spacing after section titles
\usepackage{titlesec}
%Cool maths printing
\usepackage{amsmath}
\usepackage{tabularx}
%PseudoCode
\usepackage{algorithm2e}

\titlespacing{\subsection}{0pt}{\parskip}{-3pt}
\titlespacing{\subsubsection}{0pt}{\parskip}{-\parskip}
\titlespacing{\paragraph}{0pt}{\parskip}{\parskip}
\newcommand{\figuremacro}[5]{
    \begin{figure}[#1]
        \centering
        \includegraphics[width=#5\columnwidth]{#2}
        \caption[#3]{\textbf{#3}#4}
        \label{fig:#2}
    \end{figure}
}

\lstset{
frame=single,
breaklines=true,
columns=fullflexible
}

\thiswatermark{\centering \put(-15,-100.0){\includegraphics[scale=0.3]{logo}} }
\title{\mytitle}
  \author{\myauthor\hspace{1em}\\\contact\\FWC220101    IITH-Future Wireless Communications     Assignment-1\hspace{0.5em}\hspace{0.5em}\mymodule}
\date{}
\hypersetup{pdfauthor=\myauthor,pdftitle=\mytitle,pdfkeywords=\mykeywords}
\sloppy
% #######################################
% ########### START FROM HERE ###########
% #######################################
 
 \begin{document}
 \maketitle
 \tableofcontents
    %\begin{figure}
     %   \centering
      %  \includegraphics[width=\linewidth]{}
       % \includegraphics[width=\linewidth]{}
     %\caption{\textbf{d-flip-flops.gif}}
      %  \label{fig:my_label}
    %\end{figure}
  \textbf{}{\mykeywords}
\vspace{5mm}      
\section{Abstract}
This manual shows counter constructed with 3 D FLIPFLOPS getting boolean expressionscon using   arduino in karnaugh map

\section{source question given}
counter constgructed with 3 d-flipflops input and output pairs  are named (d0,q0), (d1,q1),(d2,q2) output sequence be greycode sequence 000,001,011,010,110,111,100 repeated periodcally. combine logic expression of d1

\vspace{5mm}     
\section{flipflop way of working}
 
the single input is called data input.if the data input is high the flipflop would be SET. when the data input is low the flipflop would be RESET.that was shown in the below table.ny below table if we give high value means 1 flipflop would be SET.if we give low value means 0 flipflop would be RESET
 \begin{figure}
 
 \end{figure}
    
 \section{Transition table}
\begin{tabularx}{0.30\textwidth} {
  | >{\centering\arraybackslash}X
  | >{\centering\arraybackslash}X
  | >{\centering\arraybackslash}X | }
  \hline
 Q & Q+ & D\\
\hline
0 & 0 & 0 \\  
\hline
0 & 1 & 1 \\
\hline
1 & 0 & 0 \\
\hline
1 & 1 & 1 \\
\hline
\end{tabularx}

\section{Components}
     
       \begin{tabularx}{0.43\textwidth}{
  | >{\centering\arraybackslash}X
  | >{\centering\arraybackslash}X
  | >{\centering\arraybackslash}X | }
\hline
\textbf{Component}&\textbf{Value}& \textbf{Quantity}\\ \hline
Arduino   & UNO & 1 \\ \hline
Bread board   & - & 1 \\ \hline
Jumper wires  & M-M & 28 \\ \hline
sevensegment           & - & 1\\ \hline
Decoder      & 7447 & 1\\ \hline
Flipflop      & 7474 & 2\\ \hline
\end{tabularx}
\begin{center}
   
\end{center}

\vspace{5mm}
\subsection{Arduino}
The Arduino uno has some ground pins, analog input pins A0-A3 and digital pins D1-D13 that can be used for both input as well as output. It also has two power pins that can generate 3.3V and 5V.In the following exercises, only the ground, 5V and digital pins will be used.
   
\section{Truth table for given K-map}
\begin{tabularx}{0.50\textwidth} {
  | >{\centering\arraybackslash}X
  | >{\centering\arraybackslash}X
  | >{\centering\arraybackslash}X
  | >{\centering\arraybackslash}X
  | >{\centering\arraybackslash}X
  | >{\centering\arraybackslash}X
  | >{\centering\arraybackslash}X
  | >{\centering\arraybackslash}X 
  | >{\centering\arraybackslash}X  | }
  \hline
 X & Y & Z &X+ &Y+ & Z+ & D2 & D1 & D0\\
 0 & 0 & 0&0 & 0 & 1 &0 & 0 &1 \\  
 0 & 0 & 1&0 & 1 & 1 &0 & 1 &1\\ 
 0 & 1 & 1&0 & 1 & 0 &0 & 1 &0\\ 
 0 & 1 & 0&1 & 1 & 0 &1 & 1 &0\\ 
 1 & 1 & 0&1 & 1 & 1 &1 & 1 &1\\ 
 1 & 1 & 1&1 & 0 & 1 &1 & 0 &1\\ 
 1 & 0 & 1&1 & 0 & 0 &1 & 0 &0\\ 
 1 & 0 & 0&0 & 0 & 0 &0 & 0 &0\\ 
\hline
\end{tabularx}
\begin{center}
TABLE 1
\end{center}

\vspace{5mm}
\section{procedure}
\textbf{Step 1:} connect 5v of the Arduino to the top red of the bread board ang GND to the bottom green
\hfill \break
\hfill \break
\textbf{Step 2:} connect 2 7474 ics in  the bread board for futher connections.1 st flipflop act as 2 flipflops
\hfill \break
\hfill \break
\textbf{Step 3:} connect d13 pin in the arduino to connect breabboard the to connect the as clk to the flipflop. d13 pin connect to the pin3 and pin 11 of  1st flipflop and pin3 of 2nd flipflop 
\hfill \break
\hfill \break
\textbf{Step 4:} connect 5v to the pin 14 and pin 1 and pin4  and pin10 and pin 13 of the 1st flipflop and pin1 and pin 14 and pin4 of 2nd flipflop 
\hfill \break
\hfill \break
\textbf{Step 5:} connect 7447 decoder pins 13 to 9 to the sevensegment given a,b,c,d,e respectively and pin f to the 15 pin of sevensegment.com of the seven segment connect to vcc by using through resisitor  
\hfill \break
\hfill \break
\textbf{Step 6:} now pins 8,9,10 and 2,3,4 pins in ardino connect them in breadboard in parallel way
\hfill \break
\hfill \break
\textbf{Step 7:} take 3 new cables connect their one end in the end of the 3 parallel connected cables.other end of the cables connect to the 7,1,2 pins in the decoder serially and pin6 to the gnd
\hfill \break
\hfill \break
\textbf{Step 8:}finally run code in the arduino and check results of newmerical values shown sevensegment like 0,1,3,2,6,7,5,4 as per truthtable
\hfill \break
\hfill \break


\vspace{5mm}
\section{equation by truth table}
D2 have high logic(2,5,7,6)=sum(2,5,7,6)\\     \hfill \break
D1 have high logic(1,3,2,6)=sum(1,3,2,6)\\\hfill \break
D0 have high logic(0,1,6,7)=sum(0,1,6,7)\\
\hfill \break
\hfill \break

\section{Software}
\textbf{Execute the following code using the below provided link.}\\
\begin{center}
\fbox{\parbox{8.5cm}{\url{https://github.com/satthish-devaragatla}}}
\end{center}
 
%\begin{document}
 
 
\begin{karnaugh-map}[4][2][1][$YZ$][$X$]
        \minterms{1,2,3,6}
        \maxterms{0,4,5,7}
        %\indeterminants{2,5}
        \implicant{1}{3}
        \implicant{2}{6}
              
    \end{karnaugh-map}  
  \begin{center}
      d1--k-map diagram
  \end{center}
  
  \begin{karnaugh-map}[4][2][1][$YZ$][$X$]
        \minterms{2,5,7,6}
        \maxterms{0,1,3,4}
        %\indeterminants{2,5}
        \implicant{2}{6}
        \implicant{5}{7}
              
    \end{karnaugh-map}  
  \begin{center}
      d2--k-map diagram
  \end{center}

\begin{karnaugh-map}[4][2][1][$YZ$][$X$]
        \minterms{0,1,6,7}
        \maxterms{2,3,4,5}
        %\indeterminants{2,5}
        \implicant{1}{3}
        \implicant{2}{6}
              
    \end{karnaugh-map}  
  \begin{center}
      d0--k-map diagram
  \end{center}
\vspace{5mm}   


\section{ circuit connections}
     
        \begin{center}
\begin{tabular}{ | m{5.0em} | m{1.0cm}| m{1.0cm} |m{1.0cm} |m{0.5cm} |m{1.0cm} |m{4.0cm} |m{3.0cm} | } 
  \hline
  \textbf{Arduino} & 2,4 & 3& 8,10& 9& 13& 5V& GND \\ 
  \hline
  \textbf{7474} & 2 & 12 & 5&9 &3, 11& 1, 4, 10, 13, 14& 7\\ 
  \hline
  \textbf{7447} & 7 & 1 & & & & 16& 2, 6, 8\\ 
  \hline
\end{tabular}
\end{center}


\section{BooleanEquation}
By solving the given K-map diagram we get the boolean equation as follows \\ 
D2 = AC + BC'\\
D1 = A'C +BC'\\
D0 = A'B' +AB\\

%\end{document}

\end{document}
