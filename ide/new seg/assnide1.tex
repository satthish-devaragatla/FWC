% #######################################
% ########### FILL THESE IN #############
% #######################################
\def\mytitle{IMPLEMENTATION OF BOOLEAN LOGIC FOR D1 USING IN ARDUINO IDE}
\def\mykeywords{}
\def\myauthor{SATTHISH D}
\def\contact{sathishmahadevvarma.ds55@gmail.com}
\def\mymodule{}
% #######################################
% #### YOU DON'T NEED TO TOUCH BELOW ####
% #######################################
\documentclass[10pt, a4paper]{article}
\usepackage[a4paper,outer=1.5cm,inner=1.5cm,top=1.75cm,bottom=1.5cm]{geometry}
\twocolumn
\usepackage{graphicx}
\graphicspath{{./images/}}
%colour our links, remove weird boxes
\usepackage[colorlinks,linkcolor={black},citecolor={blue!80!black},urlcolor={blue!80!black}]{hyperref}
%Stop indentation on new paragraphs
\usepackage[parfill]{parskip}
%% Arial-like font
\usepackage{lmodern}
\renewcommand*\familydefault{\sfdefault}
%Napier logo top right
\usepackage{watermark}
%Lorem Ipusm dolor please don't leave any in you final report ;)
\usepackage{circuitikz}
\usetikzlibrary{calc}
\usepackage{tikz}

\usetikzlibrary{shapes, arrows, chains, decorations.markings,intersections,calc}
\usepackage{lipsum}
\usepackage{xcolor}
\usepackage{listings}
%give us the Capital H that we all know and love
\usepackage{float}
%tone down the line spacing after section titles
\usepackage{titlesec}
%Cool maths printing
\usepackage{amsmath}
\usepackage{tabularx}
%PseudoCode
\usepackage{algorithm2e}

\titlespacing{\subsection}{0pt}{\parskip}{-3pt}
\titlespacing{\subsubsection}{0pt}{\parskip}{-\parskip}
\titlespacing{\paragraph}{0pt}{\parskip}{\parskip}
\newcommand{\figuremacro}[5]{
    \begin{figure}[#1]
        \centering
        \includegraphics[width=#5\columnwidth]{#2}
        \caption[#3]{\textbf{#3}#4}
        \label{fig:#2}
    \end{figure}
}

\lstset{
frame=single,
breaklines=true,
columns=fullflexible
}

\thiswatermark{\centering \put(-15,-100.0){\includegraphics[scale=0.3]{logo}} }
\title{\mytitle}
  \author{\myauthor\hspace{1em}\\\contact\\FWC220101    IITH-Future Wireless Communications     Assignment-1\hspace{0.5em}\hspace{0.5em}\mymodule}
\date{}
\hypersetup{pdfauthor=\myauthor,pdftitle=\mytitle,pdfkeywords=\mykeywords}
\sloppy
% #######################################
% ########### START FROM HERE ###########
% #######################################
 
 \begin{document}
 \maketitle
 \tableofcontents
    %\begin{figure}
        \centering
        %\includegraphics[width=\linewidth]{}
        %\includegraphics[width=\linewidth]{}
    %    \caption{\textbf{Karnaugh Map}}
     %   \label{fig:my_label}
    %\end{figure}
  \textbf{}{\mykeywords}    
 \section{Abstract}
 
      This manual shows Implementation of boolean expression for d1 on using arduino after converting into D flipflop in karnaugh map
     
\section{flipflop way of working}
 
    \paragraph{FLIPFLOP way of working}
    the single input is called data input.if the data input is high the flipflop would be SET. when the data input is low the flipflop would be RESET.that was shown in the below table.ny below table if we give high value means 1 flipflop would be SET.if we give low value means 0 flipflop would be RESET
    
 \section{Transition table}
\begin{tabularx}{0.46\textwidth} {
  | >{\centering\arraybackslash}X
  | >{\centering\arraybackslash}X
  | >{\centering\arraybackslash}X | }
  \hline
 Q & Q+ & D\\
\hline
0 & 0 & 0 \\  
\hline
0 & 1 & 1 \\
\hline
1 & 0 & 0 \\
\hline
1 & 1 & 1 \\
\hline
\end{tabularx}

\section{source code given}
000--001--011--010--110--111--101--100--000---------

      \section{Components}
     
       \begin{tabularx}{0.35\textwidth} {
  | >{\raggedright\arraybackslash}X
  | >{\centering\arraybackslash}X
  | >{\raggedleft\arraybackslash}X | }
\hline
\textbf{Component} &  \textbf{Value} & \textbf{Quantity}\\
\hline
Arduino UNO &  & 1 \\  
\hline
Bread board & - & 1 \\
\hline
Jumper wires & M-M & 8 \\
\hline
Led & - & 1\\
\hline
Resistor & 150ohms & 1\\
\hline
\end{tabularx}
\begin{center}
   
\end{center}
       \subsection{Arduino} \vspace{5mm}
      The Arduino uno has some ground pins, analog input pins A0-A3 and digital pins D1-D13 that can be used for both input as well as output. It also has two power pins that can generate 3.3V and 5V.In the following exercises, only the ground, 5V and digital pins will be used.
   
\section{Truth table for given K-map}
\begin{tabularx}{0.46\textwidth} {
  | >{\centering\arraybackslash}X
  | >{\centering\arraybackslash}X
  | >{\centering\arraybackslash}X
  | >{\centering\arraybackslash}X
  | >{\centering\arraybackslash}X
  | >{\centering\arraybackslash}X
  | >{\centering\arraybackslash}X
  | >{\centering\arraybackslash}X 
  | >{\centering\arraybackslash}X  | }
  \hline
 Q2 & Q1 & Q0 & Q2+ & Q1+ & Q0+ & D2 & D1+ & D0\\
 0 & 0 & 0 &0 & 0 & 1 &0 & 0 &1 \\  
 0 & 0 & 1 &0 & 1 & 1 &0 & 1 &1\\ 
 0 & 1 & 1 &0 & 1 & 0 &0 & 1 &0\\ 
 0 & 1 & 0 &1 & 1 & 0 &1 & 1 &0\\ 
 1 & 1 & 0 &1 & 1 & 1 &1 & 1 &1\\ 
 1 & 1 & 1 &1 & 0 & 1 &1 & 0 &1\\ 
 1 & 0 & 1 &1 & 0 & 0 &1 & 0 &0\\ 
 1 & 0 & 0 &0 & 0 & 0 &0 & 0 &0\\ 
\hline
\end{tabularx}
\begin{center}
TABLE 1
\end{center}

\section{procedure}
\textbf{Step 1:} connect 5v of the Arduino to the top red of the bread board ang GND to the bottom green
\hfill \break
\textbf{Step 2:} connect d13 pin in the arduino to connect to one LED+
\hfill \break
\textbf{Step 3:} connect arduino d2 pin to the gnd or vcc according to inputs
\hfill \break
\textbf{Step 4:} connect arduino d3 pin to the gnd or vcc according to inputs
\hfill \break
\textbf{Step 5:} coonect arduino d4 pin to the gnd or vcc according to inputs
\hfill \break
\textbf{Step 6:} connect one LED+ to one end of the resisitor and other end of resistor to vcc and gnd the other terminal of LED
\hfill \break
\textbf{Step 7:}change the d2 d3 d4 pins in the arduino from vcc to gnd and observe the outputs
\hfill \break
\section{equation by truth table}
     
D1 have high logic(1,3,2,6)=sum(1,3,2,6)

\section{BooleanEquation}
By solving the given K-map diagram we get the boolean eayuation as follows : D1 = Q2Q0' + Q1Q0
\section{Software}
 Download the following code
 \begin{lstlisting}
https://github.com/satthish-devaragatla
 \end{lstlisting}
\end{document}